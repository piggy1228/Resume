\documentstyle{article}
\oddsidemargin=.05in
\evensidemargin=.05in
\textwidth=6.3in
\topmargin=-1.0in
\textheight=9in
\parindent=0in
\pagestyle{empty}

\newenvironment{list1}{
  \begin{list}{\ding{113}}{%
      \setlength{\itemsep}{0in}
      \setlength{\parsep}{0in} \setlength{\parskip}{0in}
      \setlength{\topsep}{0in} \setlength{\partopsep}{0in}
      \setlength{\leftmargin}{0.17in}}}{\end{list}}
\newenvironment{list2}{
  \begin{list}{$\bullet$}{%
      \setlength{\itemsep}{0in}
      \setlength{\parsep}{0in} \setlength{\parskip}{0in}
      \setlength{\topsep}{0in} \setlength{\partopsep}{0in}
      \setlength{\leftmargin}{0.2in}}}{\end{list}}

\vspace*{-.05in}
\begin{document}

\begin{center}
{\Large Kexin Zhu} \\[.5pc]

2930 Chesstnut St Philadelphia, PA 19104\\
(518)833-4773 $|$ zhukexin951228@gmail.com $|$ http://kexinzhu.com  \\
\end{center}



{\large \bf Education} \\*[-.8pc]
\underline{\hspace{6.3in}} 

{\bf University of Pennsylvania \hfill \it May 2020 }\\
\rm Master of Science in Engineering:  Computer $\&$ Information Science \rm \\
Cumulative GPA: 3.53/4.0\rm\\
{\bf Rensselaer Polytechnic Institute \hfill \it May 2018 }\\
Bachelor of Science:  Computer Science  dual  Mathematics \rm \\
\rm Cumulative GPA: 3.97/4.0  \quad {(Summa Cum laude)} \\
%

%{\bf Xi'an Jiaotong-Liverpool University \hfill \it 2013-2015 }\\
%Transferred Out, Major: \bf Financial Mathematics \rm \\
%Cumulative GPA: \bf 3.81/4.0 \rm  (\it WES verified\rm)\\



{\large \bf Coursework/Skills} \\*[-.8pc]
\underline{\hspace{6.3in}} 
\bf Advanced Mathematics\& Statistics: \rm  Linear Algebra, Mathematical Statistics, Mathematical Models of Operation Research, Computational Optimization\\
\bf Advanced Computer Science: \rm  Software Systems, Database $\&$ Info Systems, Big Data Analytics, Operating System, Machine Learning, Intro to Artificial Intelligence , Programming Language, Natural Language Processing\\
\bf Programming: \rm Proficient in C/C++, Python, SQL, Java, Matlab,  MongoDB, Neo4j, R, Haskell, Prolog\\
%\bf Tool: \rm Oracle, MongoDB, AWS, Google Cloud, Docker, Jupyter, git\rm 

{\large \bf Selected Projects} \\*[-.8pc]
\underline{\hspace{6.3in}} \\
\bf PennCloud (C/C++)\hfill{\it Oct-Dec. 2018}\rm\\
\vspace{-4mm}
\begin{list2}
\item Built a cloud platform with its webmail and storage service, analogous to Gmail and Google Drive
\item A set of frontend servers can be accessed with browsers and users are able to interact with the services
\item  A distributed storage system in the backend, home to storage of all state and key-value store abstractions, is built to guarantee the consistency of storage by achieving load balancing and fault tolerance
\end{list2}

\bf  Food for thought (Full stack)\hfill{\it Oct-Dec. 2018}\rm\\
\vspace{-4mm}
\begin{list2}
\item Developed a webpage which recommends Airbnb homes in NYC to users based on their food 
preference and food options in the neighborhoods, with data from Airbnb and Yelp, stored on cloud hosting (AWS) 
\item Users are able to customize their home preferences, including price range and choice of neighborhood 
\item Results are shown on Google Maps, where users are able to be directed to webpages of  the Airbnb homes, as well as nearby restaurants. Altogether, users can integrate food and homes into plans,  and bookmark them in their accounts
\end{list2}

\bf  Image-News Matching System (Python)\hfill{\it Feb-May. 2018}\rm\\
\vspace{-4mm}
\begin{list2}
\item Developed a system that can be used to match news story and news photos
\item Used TF-IDF (Term Frequency Inverse Document Frequency) to extract best selections of key phrases to describe the news and
\item Matched the results of photos and news stories via image captioning, as well as key phrase extraction  
\end{list2}
%\bf  Pac-man Projects (Python)\hfill{\it Jan-May. 2018}\rm\\
%\vspace{-4mm}
%\begin{list2}
%\item Designed agents for the classic version of Pacman, including ghosts
%\item Implemented search, knowledge representation and reasoning, planning, reasoning under uncertainty, and machine learning knowledge to design the agents algorithm
%\end{list2}

%\bf Process Simulation Framework (Python)\hfill{\it Oct-Nov. 2017}\rm\\
%\vspace{-4mm}
%\begin{list2}
%\item Implemented a rudimentary simulation of an operating system
%\item Simulated three stages of process: ready, running and blocking according to three algorithms: First Come, First Served, Shortest Remaining Time and Round Robin; to get the CPU status
%\end{list2}

\bf Image Inpainting (Matlab)\hfill{\it Feb-May. 2017}\rm\\
\vspace{-4mm}
\begin{list2}
\item Reconstructed contaminated images with ADMM (Alternating Direction 
Method of Multipliers) 
\item Transformed a contaminated image with characters on it to a full image using computational optimization to recover pixels in the characters\\
\end{list2}


%{\large \bf Intern} \\*[-.8pc]
%\underline{\hspace{6in}} 
%{\bf Oracle (China) Software Systems Co., Ltd. \hfill \it  Beijing, May.-July. 2017}\\
%\rm Database, JDBC and Project Management, \it Intern \rm\\
%\vspace{-4mm}
%\begin{list2}
%\item Participate AMURSKY Gas Processing Plant project of Gazprom and design the Transmittals in Primavera Unifier system.
%\item Finish the project planning by mapping out the schematic schedule of a 1,000 KTA ethylene plant and project costing.
%\end{list2}


{\large \bf Research} \\*[-.8pc]
\underline{\hspace{6.3in}} 
%{\bf Queueing Theory (Python)}\hfill {\it Aug.-Dec. 2017}\\
%\vspace{-4mm}
%\begin{list2}
%\item Induce and compare different queueing principles.
%\item  Code a program to generate a retail store model and calculate customer's average waiting time under different queueing principles. 
%\item Compare the results and generate the best solution under different circumstances.
%\end{list2}
%{\bf Duke Kunshan University \hfill \it  July-Aug. 2018}\\
%\rm \it Iris Recognition, Intern \rm\\
{\bf Iris Recognition (Python)}\hfill {\it July-Aug. 2018}\\
\vspace{-4mm}
\begin{list2}
\item Given an eye image, extracted the iris section and compared with original iris data to verify user identity
\item Used traditional computer vision method to recognize and capture the Iris from eye images
\item Removed noises, such as reflection in pupils and glasses by distinguishing relative pixel color\\
\end{list2}

%{\bf Circadian Rhythms (R)}\hfill{\it July-Aug. 2017}\\
%\vspace{-4mm}
%\begin{list2}
%\item Analyzed three mouse genes' performance in forty-eight hours under control experiment 
%\item Wrote a bootstrapping R program to observe the same parameter properties of the three mice
%\item Obtained the sampling distribution and confidence interval to analyze the circadian rhythms of mice\\
%\end{list2}

{\large \bf Professional Experience} \\*[-.8pc]
\underline{\hspace{6.3in}} 
Undergraduate Teaching Assistant:  Intro to Algorithm, Mathematical Statistics, Foundations of Computer Science and Beginning Prog for Engineers \\

%Mentor for \bf Beginning Prog For Engineers (CSCI 1190)\rm\\

%{\large \bf Patent} \\*[-.8pc]
%\underline{\hspace{6in}} \\
%\\
%{\bf ``Sliding Desks \& Chairs without Noise"}\\
% \textbf{ZL 200620076549.X}, Jan. 9th 2008\\


%{\large \bf Activities} \\*[-.8pc]
%\underline{\hspace{6in}} \\
%\\
%{\bf Overseas Volunteer\rm, \it Activating the Leadership Potential of Young People \hfill \it Jan. 2015}

%\begin{list2}
%\item Spread knowledge about the danger and prevention of AIDS.
%\end{list2}
\end{document}